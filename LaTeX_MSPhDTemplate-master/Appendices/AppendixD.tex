\cleartooddpage[\thispagestyle{empty}]
\chapter{Cellular Biology}

\section{TUNEL-assay}\label{APPENDIXD_TUNEL}
Terminal deoxynucleotidyl transferase dUTP-biotin nick end labeling (TUNEL)is an assay for detecting DNA fragementation: an aspect of cellular damage and apoptosis. TUNEL uses the enzynme terminal deoxynucleotidyl transferace (TdT) to attach labeled deoxyuridine triphosphate (dUTP) onto the 3'-hydroxyl termini of internucleosomal DNA fragmentation. Modification of dUTP through the addition of fluorphores or haptens, such as biotin, allow for DNA fragments to be detected directly using a fluorescently-modified nucleotide and fluorescence microscopy or flow cytometry.


\section{VCAM-1}\label{APPENDIXD_VCAM1}
VCAM-1 is a member of the immunoglobulin superfamily (cell surface and soluble proteins involved in the recognition and/or binding of cells) and encodes a cell surface sialoglycoprotein (sialic acid and glycoprotein combination) expressed by cytokine-activated endothelium. This membrane protein acts as a ligand for leukocyte-endothelial cell adhesion, signal transduction, and may play a role in the development of artherosclerotic and/or inflammatory based pathologies. Molecules containing VCAM-1 counterreceptors (VLA-4 on monocytes and lymphocytes) can adhere to VCAM-1 activated cells\cite{kaufmann2007}. Bound leukocytes may undergo polarized motility into the vascular wall, disrupting the cellular and matrix components of the vasuclautre, and degrading endothelial cell permeability. 