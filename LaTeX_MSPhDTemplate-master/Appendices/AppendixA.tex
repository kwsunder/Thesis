\cleartooddpage[\thispagestyle{empty}]
\chapter{Statistics}\label{APPENDIXA}

In this type of predictive modeling, there exists an input-output dataset (X,Y) $ \in X x Y$ w with an unknown probability distribution \textit{P}. The goal of predictive modeling is to find a function $f_n : X \rightarrow Y$, that is determined using a training set (X\textsubscript{1}, Y\textsubscript{1},...,(X\textsubscript{n}, Y\textsubscript{n}) of \textit{n} random pairs distributed as (X,Y). A desirable solution of \textit{f}\textsubscript{n} is one that, given a new data-point $x \in X$, the resultant \textit{f}\textsubscript{n}(x) is an accurate prediction of the true output $y \in Y$. This desired outcomes not only relies on the chosen function's predictive accuracy, but also of the selecting of relevant variables that are capable of achieving desired predictions. For desired models, it is often preferred to find the prediction function that achieves the desired accuracy while using the minimal amount of variables required: i.e a \textit{parsimonious} model. Brute-force methods of testing all variable combinations becomes increasingly unviable, especially when the number of variables in a dataset is larger than the number of \textit{n} data points (cases) available for analysis: often refereed to the "large \textit{p}, small \textit{n} paradigm". One type of methodology to determine a desired model is through the use of sparsity-based regularization methods \cite{tibshirani1996regression,frank1993bridge,tibshirani2002diagnosis,zou2005elastic}

\section{Section 1}\label{Mult_log_regression}
Multiple logistic regression (MLR) analysis looks both to estimate the odds of a dichotomous outcome occurring, and to determine the impact of an individual variable (covariate) in relation to the other covariates in a model. The probability of an outcome occurring in MLR can be calculated as such:

\begin{equation}
\hat{p} = \frac{exp(b_0 + b_1 X_1 + b_2 X_2 + ... + b_p X_p)}{1 + exp(b_0 + b_1 X_1 + b_2 X_2 + ... + b_p X_p)}
\end{equation}

\^{p} being the probability of the desired outcome, X\textsubscript{1} through X\textsubscript{p} as the individual dependent variables applied to the model, and b\textsubscript{1} to b\textsubscript{p} being each variable's (respective) regression coefficients. 
To determine the expected log odds ratios of the model's variables, the $logit$ function of the above equation can be calculated:

\begin{equation}
\begin{aligned}
logit[\hat{p}] & = ln[\frac{\hat{p}}{1-\hat{p}}] \\
			   & = ln\left[\frac{\frac{exp(b_0 + b_1 X_1 + b_2 X_2 + ... + b_p X_p)}{1 + exp(b_0 + b_1 X_1 + b_2 X_2 + ... + b_p X_p)}}{1 - \frac{exp(b_0 + b_1 X_1 + b_2 X_2 + ... + b_p X_p)}{1 + exp(b_0 + b_1 X_1 + b_2 X_2 + ... + b_p X_p)}}\right] \\
			   & = ln\left[\frac{\frac{exp(b_0 + b_1 X_1 + b_2 X_2 + ... + b_p X_p}{1 + exp(b_0 + b_1 X_1 + b_2 X_2 + ... + b_p X_p}}{\frac{1}{1 + exp(b_0 + b_1 X_1 + b_2 X_2 + ... + b_p X_p}}\right] \\
			   & = ln [ exp(b_0 + b_1 X_1 + b_2 X_2 + ... + b_p X_p] \\ 
			   & = b_0 + b_1 X_1 + b_2 X_2 + ... + b_p X_p
\end{aligned}
\end{equation}

Taking the $logit$ of the desired outcome's probability, transforms the occurrence of the event given Xs into a simplified linear function.

For each variable added to a regression model, the resultant R\textsuperscript{2} (coefficient of multiple determination) may increase, indicating an improved fit of the data. However applying a large number of variables to a predictive model may result in over-fitting without a significantly large dataset: large \textit{p}, small \textit{n} paradigm. In such an event, the R\textsuperscript{2} values, regression coefficients, and any statistical significance (\textit{p}-values) determined may be misleading. To reduce the initial choices of variables in assessed predictive models, the correlation between variables were determined. The correlation of data can be determine by:

\begin{equation}
r_{jk} = \frac{s_{jk}}{s_j s_k} = \frac{\sum_{i=1}^{n}(x_{ij} - \overline{x}_j)(x_{ik} - \overline{x}_k)}{\sqrt{\sum_{i=1}^{n}(x_{ij} - \overline{x}_j)^2}{\sqrt{\sum_{i=1}^{n}(x_{ik} - \overline{x}_k)^2}}}
\end{equation}
with r as the Pearson correlation coefficient between variables x\textsubscript{j} and x\textsubscript{k}, n as the sample size, and $\overline{x}$ is a variable sample mean. Correlations between the variables are often displayed via a correlation table:

\[
R = 
\begin{bmatrix}
    1       & r_{12} & r_{13} & \dots  & r_{1p} \\
    r_{21}  & 1      & r_{23} & \dots  & r_{2p} \\
    r_3{1}  & r_{32} & 1      & \dots  & r_{3p} \\
    \vdots  & \vdots & \vdots & \ddots & \dots  \\
    r_{p1}       & r_{p2}     & r_{p3} & \dots & 1
\end{bmatrix}
\]

Initial correlation analysis of all available geometric and hemodynamic variables was performed to eliminate highly correlated variables from analysis: i.e aneurysm volume and surface area are highly correlated so surface area was removed from analysis. 

From the remaining variables, stepwise MLR was implemented to determine the parsimonious model. In stepwise regression, a linear regression is first performed for each variable X one at a time, and the variable with the highest R\textsuperscript{2} is kept for the model. Next, a multiple regression step is performed with the kept variable and each remaining variable. The variable with the largest increase in R\textsubscript{2}, if the \textit{p} value of the R\textsuperscript{2} is below a desired cuttoff ($<$0.05), is added to the model. The calculation of the \textit{p} value of an increase in R\textsuperscript{2} resulting from the increasing of X variable(s) from \textit{a} to \textit{b} is as follows:
\begin{equation}
p_{ab} = \frac{(R^2 _b - R^2 _a)/(b-a)}{(1-R^2 _b) / (n-b-1)}
\end{equation}
with the total sample size \textit{n}.

Each time a new variable is added to the model, the impact of removing any of the other variables (already added to the model) on outcomes is tested. The chosen (removed) variable is excluded from the model if it does not make R\textsuperscript{2} significantly worse. This process is continued till adding any new variables does not increase R\textsuperscript{2} and removing any X variables does not significantly decrease R\textsuperscript{2}


In the event that all of the independent variables in the model are completely uncorrelated with each other, the interpretation of coefficients are as such:
\begin{equation}
OR = exp(b_1)^z
\end{equation}
Where z is the number of unit changes for a variable X, and OR is the odds ratio resultant from said change. When the variables are not uncorrelated, the $OR = exp^zb_1$ is expressed as the change of unit z for a variable \textit{adjusted in relation to the impacts of the other variables in the model}. This stresses the need to assess collinearity between variables prior to model assessment.


   in 
Section \ref{APPENDIXA_SECTION1}.

\section{Section 2}\label{Nearest Shrunken Centroid}
Limitations may arise in applying multiple logistic regression analysis to data sets with a large number of variables in relation to the number of samples.

The Nearest Shrunken Centroid (NSC) method is a statistical classification methodology developed by Tibshirani et al.\cite{tibshirani2002diagnosis} as a means to improve predictions in high-dimensional data. Additionally, a 2014 study by Finch \cite{finch2014comparison} compared a number of methods for statistical group prediction. The NSC method was found to be robust in terms of accuracy and identification of predictor variables over other methods when dealing with high-dimensional datasets. Initially developed for predictions within genetic data, NSC aims to shrink class (e.g. rupture status) centroids towards the \textit{overall} centroid of they entire data-set after standardizing by the within-class deviation for each variable. This standardization of the resultant centroid gives higher impact to variables whose expression is more stable withing samples of the same class. Due to the number of  geometric and hemodynamic variables that may impact IA rupture, the NSC method was investigated as a useful methodology to help predict rupture potential within our data-set.

In our study, $x_{ijk}$ represents the value of quantified variables $i = 1, ..., p$ in aneurysm $j = 1,...,n_k$ of class \textit{k}. The mean for variable \textit{i} in each class \textit{k} is calculated

\begin{equation}
\label{variable_mean_NSC}
\overline{x}_{ik} = \sum_{j=1}^{n_k}\frac{x_{ijk}}{n_k}
\end{equation}
and the \textit{overall} mean for variable \textit{i} is calculated.

\begin{equation}
\label{overall_variable_mean_NSC}
\overline{x}_i =  \sum_{k=1}^{K} \sum_{j=1}^{n_k}\frac{x_{ijk}}{n_k}
\end{equation}
A \textit{t}-statistic value, $d_{ik}$, for each variable is calculated, comparing class \textit{k} to the overall mean:

\begin{equation}
\label{d_for_NSC}
d_{ik} = \frac{\overline{x}_{ik} - \overline{x}_i}{m_k \cdot (s_i + s_0)}
\end{equation}
with

\begin{equation}
\label{s2_for_NSC}
s^{2}_{i} = \frac{1}{n-K}\sum_{k=1}^{K}\sum_{j=1}^{n_k}(x_{ij} - \overline{x}_{ik})^2
\end{equation}
and

\begin{equation}
\label{m_for_NSC}
m_k = \sqrt{1/n_k + 1/n}, \text{ where } n = \sum_{k=1}^{K}n_k
\end{equation}
The $s_0$ in (\ref{d_for_NSC}) is used as a \textit{regularization parameter} to help protect form large $d_{ik}$ values occurring from variables at low expression levels. The value of $s_0$ is determined as the median value of $s_i$ over the set of variables. 

With the inclusion of $d_{ik}$, the class centroid can be rewritten as:

\begin{equation}
\label{xik_with_d_NSC}
\overline{x}_{ik} = \overline{x}_i + m_k(s_i+S_0)d_{ik}
\end{equation}
The NSC method shrinks each $d_{ik}$ toward zero, creating a new value $\d'_{ik}$ which generates a new shrunken centroid value

\begin{equation}
\label{shrunken_xik_for_NCS}
\overline{x}'_{ik} = \overline{x}_i + m_k(s_i+s_0)d'_{ik}
\end{equation}
The value of shrinkage is determined through \textit{soft thresholding}, where the absolute value of $d_{ik}$ is reduced by $\Delta$ and is given the value of 0 if the result is $<0$, with

\begin{equation}
\label{soft_thresholding_NSC}
d'_{ik} = sign(d_{ik})(|d_{ik}| - \Delta)_+
\end{equation}

This combination of shrinkage and \textit{soft thresholding} can result in many 'noisy' $\overline{x}_{ik}$ being close to the value of the overall mean $\overline{x}_i$. If $\Delta$ shrinks $d_{ik}$ to zero for all \textit{k}, then the centroid for variable \textit{i} is the same for all classes and is excluded from prediction analysis. 

The value of $\Delta$ is determined through cross-validation: fitting the model for many values of $\Delta$ and determining the level of error per chosen $\Delta$. The $\Delta$ resulting in the smallest error was chosen for our prediction.



\section{Section 3}\label{Elasitc Net Regularization}
Elastic Net Regularization (ENR) overcomes some of the limitations of the LASSO selection method, primarily being able to accurately handle data sets with a high number of variables in relation to the sample size \cite{tibshirani1996regression,efron2004least}. Additionally, the ENR method is able to handle data sets with groups of highly correlated variables. 

ENR solves two optimization problems:
\begin{equation} \label{ENR_optimization}
\begin{aligned}
&\tilde{\beta} = \text{arg min}_\beta \sum_{i=1}^{N} (y_i - (X\beta)_i)^2 \\
&\text{subject to} \sum_{j=1}^{p} |\beta_j | \le t_1 \text{ and } \sum_{j=1}^{p} \beta_j^2 \le t_2
\end{aligned}
\end{equation}
were a penalty is places on the \textit{L}\textsubscript{i} norm$(\sum_{j=1}^{p} |\beta_j|)$ and the \textit{L}\textsubscript{2} norm $(\sum_{j=1}^{p}\beta_{j}^{2} \le t_2$ of the regression coefficients. The purpose of these penalties are as follows: \textit{L}\textsubscript{1} performs variable selection by setting some coefficients to 0, and \textit{L}\textsubscript{2} works toward group selection by shrinking the coefficients of correlated variables toward each other. Re-writing equation \ref{ENR_optimization} in the Lagrangian form using two tuning parameters $(lambda_1 \text{ and } \lambda_2)$ is as follows:
\begin{equation} \label{ENR_lagrangian}
\tilde{\beta} = \text{arg min}_\beta \left(\sum_{i=1}^{N} (y_i - (X\beta)_i)^2 + \lambda_1 \sum_{j=1}^{p} |\beta_j| + \lambda_2 \sum_{j=1}^{p} \beta_{j}^{2}\right)
\end{equation}

The choice of tuning parameter values is performed by analyzing an array of $\lambda_2$ values (0, 0.01, 0.1, 1, 10, and 100). For each value in the array, the LARS-EN algorithm calculates the resultant $\lambda_1$ value. The $lambda_1$ value that yields the smallest \textit{k}-fold cross validation error, and its $lambda_2$ value used to generate it, are used as the tunning parameters for the ENR method.


\section{Section 4}\label{Reciever Operator Characterisitics}
To assess the diagnostic ability of predictive model(s), a receiver operating characteristic curve (ROC) is often deployed (REFERENCES). The ROC curve assesses a model's predictive true positive rate (TPR) against its false positive rate (FPR) as a means to determine overall predictive strength (HANLEY). From a statistical perspective, ROC analysis can be considered as a plot of the power (probability of a test correctly rejecting the null hypothesis when an alternative hypothesis is true) 

\begin{equation}
\begin{aligned}
& TPR = \frac{\Sigma True Positive}{\Sigma Condition Positive} \\
& FPR = \frac{\Sigma False Ppositive}{\Sigma Condition Negative} \\
& FNR = \frac{\Sigma False Negative}{\Sigma Condition Positive} \\ 
& Specificity = \frac{\Sigma True Negative}{\Sigma Condition Negative} 
\end{aligned}
\end{equation}
When dealing with a binary classification, as per this study, the predictive test measure for  each instance is denoted by a continuous random variable (x). Given a desired threshold (T), each instance is positive if x$>$T and negative if x$<$T. Setting the probability distribution functions of the positive and negative values of x to f\textsubscript{p}(x) and f\textsubscript{n}(x) respectively, the . Given this, TPR is calculated as:

\begin{equation}
TPR(T) = \int_{T}^{\infty} f_p(x) dx
\end{equation}

and the FNR as:
 
\begin{equation}
FPR(T) = 1- \int_{T}^{\infty} f_n(x) dx
\end{equation}

The ROC curve is generated by plotting TPR(T) against FPR(T) parametrically, varying across T, or as a plot of:

\begin{equation}
ROC(T) = 1 - f_p(f_n ^{-1} (1-T))
\end{equation}

over T from [0,1] where $f_p$\textsuperscript{-1}(1-T) = inf

Comparing the resultant ROC curves across multiple models provides the selection of the desired model based off of varying predictive accuracies. To quantify the predictive accuracy, the area under the curve (AUC) of the ROC curve is calculated, as it equals the probability of a classifier ranking a positive instance higher than a negative instance (both chosen at random).

\begin{equation}
\begin{aligned}
A &= \int_{\infty}^{-\infty} TPR(T) FPR' (T) dT \\
  &= \int_{-\infty}^{\infty}\int_{-\infty}^{\infty} I(T' >T)f_1 (T' )f_0 (T) dT' dT = P(X_1 > X_0)
\end{aligned}
\end{equation}

The initial integral has reversed boundaries due to larger T values having a lower value on the x-axis.
  
  


\section{Section 5}\label{APPENDIXA_SECTION2}

Docendi eligendi sit et, pri ea dicam eligendi percipitur, has soleat
dolores convenire te. Sed altera placerat an, id verterem abhorreant
interesset mea. Eum at ceteros efficiantur. Eos id voluptaria efficiendi
comprehensam.

In mel modo dicam vocibus, eruditi consectetuer vim no, cu quaestio
instructior eum. Justo nostrud fuisset ea mea, eam an libris repudiandae
vituperatoribus. Est choro corrumpit definitionem at. Vel sint adhuc vocibus
ea, illud epicuri eos no. Sea simul officiis ea, et qui veri invidunt
appellantur. Vix et eros ancillae pertinax.
